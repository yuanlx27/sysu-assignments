\documentclass[a4paper]{article}
\usepackage[UTF8]{ctex}
\usepackage{geometry}
\usepackage{listings}
\usepackage{xcolor}
\usepackage{graphicx}
\usepackage{amsmath}
\usepackage{float}
\usepackage{hyperref}

\geometry{a4paper, scale=0.8}

\lstset{
    language=C++,
    basicstyle=\ttfamily\footnotesize,
    keywordstyle=\color{blue},
    commentstyle=\color{gray},
    stringstyle=\color{red},
    breaklines=true,
    frame=single,
    numbers=left,
    numberstyle=\tiny\color{gray},
    tabsize=4
}

\title{计算机图形学 实验报告三:Phong Shading 与 VBO}
\author{姓名:元朗曦 \quad 学号:23336294}
\date{\today}

\begin{document}

\maketitle

\section{实验目的}
\begin{enumerate}
    \item 理解并掌握 Phong 光照模型(Phong Shading)的原理与实现方法。
    \item 掌握 OpenGL 中顶点缓冲对象(VBO, Vertex Buffer Object)的使用,理解其相对于立即模式(Immediate Mode)的优势。
    \item 不依赖第三方辅助库(如 GLM, GLAD),手动实现必要的数学运算库(向量与矩阵)及 OpenGL 函数加载,深入理解底层机制。
    \item 使用 Dear ImGui 构建简单的交互界面,实时调整光照参数及渲染模式。
\end{enumerate}

\section{实验原理}

\subsection{Phong 光照模型}
Phong 光照模型是计算机图形学中常用的局部光照模型,它将物体表面的光照分为三个部分:环境光(Ambient)、漫反射(Diffuse)和镜面反射(Specular)。

\begin{equation}
    I = I_{ambient} + I_{diffuse} + I_{specular}
\end{equation}

\begin{itemize}
    \item \textbf{环境光}:模拟场景中来自四面八方的散射光,通常是一个常量。
    \[ I_{ambient} = k_a \cdot L_a \]
    \item \textbf{漫反射}:模拟光线照射到粗糙表面后的散射。强度与光线方向和法线方向的夹角余弦成正比(Lambert 定律)。
    \[ I_{diffuse} = k_d \cdot L_d \cdot \max(\vec{N} \cdot \vec{L}, 0) \]
    其中 $\vec{N}$ 为法线向量,$\vec{L}$ 为指向光源的向量。
    \item \textbf{镜面反射}:模拟光线照射到光滑表面产生的高光。强度与视线方向和反射光线方向的夹角余弦的幂次成正比。
    \[ I_{specular} = k_s \cdot L_s \cdot \max(\vec{V} \cdot \vec{R}, 0)^{\alpha} \]
    其中 $\vec{V}$ 为指向观察者的向量,$\vec{R}$ 为反射向量,$\alpha$ 为高光系数(Shininess)。
\end{itemize}

\subsection{顶点缓冲对象 (VBO)}
VBO 是 OpenGL 提供的一种机制,允许将顶点数据(位置、法线、纹理坐标等)存储在显卡的高速显存中。
\begin{itemize}
    \item \textbf{优势}:相比于立即模式(`glBegin`/`glEnd`)需要在每一帧通过 CPU 向 GPU 发送数据,VBO 只需要在初始化时发送一次数据,渲染时 GPU 直接从显存读取,极大地减少了 CPU-GPU 之间的带宽消耗,提高了渲染效率。
\end{itemize}

\section{实验内容}

\subsection{项目环境搭建}
本项目使用 CMake 构建,通过 `FetchContent` 模块自动拉取 GLFW 和 Dear ImGui 库。为了深入理解底层,本项目\textbf{未使用} GLAD(OpenGL 加载库)和 GLM(数学库),而是手动实现了相关功能。

\subsection{核心代码实现}

\subsubsection{1. OpenGL 函数加载}
由于未使用 GLAD,需要手动定义函数指针并使用 `glfwGetProcAddress` 加载 OpenGL 核心函数。
\begin{lstlisting}
// Define function pointer types
typedef void (APIENTRY *PFNGLGENBUFFERSPROC) (GLsizei n, GLuint *buffers);
// ... other type definitions ...

// Declare function pointer variables
PFNGLGENBUFFERSPROC glGenBuffers = NULL;
// ...

// Load functions
void loadOpenGLFunctions() {
    glGenBuffers = (PFNGLGENBUFFERSPROC)glfwGetProcAddress("glGenBuffers");
    // ... load other functions ...
}
\end{lstlisting}

\subsubsection{2. 简易数学库实现}
实现了 `Vec3` 和 `Mat4` 结构体,支持向量加减乘除、点乘、叉乘、归一化,以及矩阵的平移、旋转、透视投影和 LookAt 变换。
\begin{lstlisting}
struct Vec3 {
    float x, y, z;
    // ... operator overloads ...
    Vec3 normalize() const { /* ... */ }
    float dot(const Vec3& other) const { /* ... */ }
    Vec3 cross(const Vec3& other) const { /* ... */ }
};

struct Mat4 {
    float m[4][4];
    static Mat4 perspective(float fov, float aspect, float near, float far) { /* ... */ }
    static Mat4 lookAt(Vec3 eye, Vec3 center, Vec3 up) { /* ... */ }
    static Mat4 rotate(float angle, Vec3 axis) { /* ... */ }
    // ...
};
\end{lstlisting}

\subsubsection{3. 球体几何生成}
通过经纬度划分法生成球体的顶点位置和法线数据。
\begin{lstlisting}
void generateSphere(float radius, int stacks, int slices) {
    for (int i = 0; i < stacks; ++i) {
        for (int j = 0; j < slices; ++j) {
            // Compute spherical coordinates (x, y, z)
            // Compute normal (normalized position vector)
            // Push data into std::vector<float> sphereVertices
            // Each vertex contains 6 floats: x, y, z, nx, ny, nz
        }
    }
}
\end{lstlisting}

\subsubsection{4. Shader 实现 (Phong Shading)}
\textbf{Vertex Shader}: 计算世界坐标系下的顶点位置和法线,并传递给 Fragment Shader。
\begin{lstlisting}
#version 330 core
layout (location = 0) in vec3 aPos;
layout (location = 1) in vec3 aNormal;

out vec3 FragPos;
out vec3 Normal;

uniform mat4 model;
uniform mat4 view;
uniform mat4 projection;

void main() {
    FragPos = vec3(model * vec4(aPos, 1.0));
    Normal = mat3(transpose(inverse(model))) * aNormal;  
    gl_Position = projection * view * vec4(FragPos, 1.0);
}
\end{lstlisting}

\textbf{Fragment Shader}: 在片元阶段计算 Phong 光照。
\begin{lstlisting}
#version 330 core
out vec4 FragColor;
in vec3 Normal;
in vec3 FragPos;

uniform vec3 lightPos; 
uniform vec3 viewPos; 
uniform vec3 lightColor;
uniform vec3 objectColor;
uniform float shininess;

void main() {
    // Ambient
    float ambientStrength = 0.1;
    vec3 ambient = ambientStrength * lightColor;
  	
    // Diffuse 
    vec3 norm = normalize(Normal);
    vec3 lightDir = normalize(lightPos - FragPos);
    float diff = max(dot(norm, lightDir), 0.0);
    vec3 diffuse = diff * lightColor;
    
    // Specular
    float specularStrength = 0.5;
    vec3 viewDir = normalize(viewPos - FragPos);
    vec3 reflectDir = reflect(-lightDir, norm);  
    float spec = pow(max(dot(viewDir, reflectDir), 0.0), shininess);
    vec3 specular = specularStrength * spec * lightColor;  
        
    vec3 result = (ambient + diffuse + specular) * objectColor;
    FragColor = vec4(result, 1.0);
} 
\end{lstlisting}

\subsubsection{5. 渲染循环与 VBO/立即模式切换}
在主循环中,根据 ImGui 的复选框状态选择渲染方式。
\begin{lstlisting}
if (useVBO) {
    glBindVertexArray(VAO);
    glDrawArrays(GL_TRIANGLES, 0, sphereVertexCount);
} else {
    // Immediate mode
    glBindVertexArray(0);
    glBegin(GL_TRIANGLES);
    for (int i = 0; i < sphereVertexCount; ++i) {
        // Manually pass normals and vertices
        glVertexAttrib3f(1, ...); // normal
        glVertex3f(...);          // position
    }
    glEnd();
}
\end{lstlisting}

\section{实验结果}

\subsection{运行效果}
程序运行后,显示一个被光照照亮的球体。

\begin{figure}[H]
    \centering
    \includegraphics[width=0.8\textwidth]{assets/images/demo.png}
    \caption{Phong 光照模型渲染效果}
    \label{fig:demo}
\end{figure}

\begin{itemize}
    \item 球体表面呈现出平滑的光照过渡,可以看到明显的高光(Specular)、漫反射(Diffuse)和环境光(Ambient)效果。
    \item 球体在不断旋转,展示了动态的光照变化。
\end{itemize}

具体效果见 \texttt{assets/demo.mp4}。

\subsection{交互控制}
通过 ImGui 面板,可以实时调整以下参数:
\begin{itemize}
    \item \textbf{Use VBO}: 切换 VBO 渲染和立即模式渲染。
    \item \textbf{Light Position}: 调整光源在空间中的位置。
    \item \textbf{Light Color}: 调整光源颜色。
    \item \textbf{Object Color}: 调整球体材质颜色。
    \item \textbf{Shininess}: 调整高光系数,改变高光的大小和锐度。
\end{itemize}

\subsection{性能与效果对比}
\begin{enumerate}
    \item \textbf{视觉效果}:无论是使用 VBO 还是立即模式,视觉上渲染出的球体和光照效果是完全一致的,因为底层的 Shader 计算逻辑相同。
    \item \textbf{代码逻辑}:
    \begin{itemize}
        \item \textbf{VBO}:代码简洁,只需在初始化时绑定数据,绘制时调用 `glDrawArrays`。
        \item \textbf{立即模式}:需要在每一帧的循环中遍历所有顶点,调用大量的 `glVertex` 函数,代码冗长且效率低。
    \end{itemize}
    \item \textbf{效率分析}:虽然在本实验简单的球体模型下帧率差异可能不明显,但在处理复杂模型(数万个顶点以上)时,VBO 的性能将显著优于立即模式,因为减少了 CPU 到 GPU 的函数调用开销和数据传输瓶颈。
\end{enumerate}

\section{实验总结}
本次实验成功实现了基于 Phong 光照模型的球体渲染,并对比了 VBO 和立即模式两种渲染方式。
\begin{enumerate}
    \item 深入理解了图形管线中 Vertex Shader 和 Fragment Shader 的协作流程,特别是法线变换和光照计算的细节。
    \item 掌握了不依赖 GLM 和 GLAD 库进行 OpenGL 开发的方法,对 OpenGL 的状态机机制和函数指针加载有了更直观的认识。
    \item 验证了 VBO 作为现代 OpenGL 核心特性的重要性,它是实现高效实时渲染的基础。
    \item 通过 ImGui 实现了参数的可视化调节,增强了实验的交互性和直观性。
\end{enumerate}

\end{document}
