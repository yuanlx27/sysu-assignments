\documentclass[a4paper,12pt]{article}
\usepackage[UTF8]{ctex}
\usepackage{graphicx}
\usepackage{geometry}
\usepackage{hyperref}
\usepackage{listings}
\usepackage{xcolor}
\usepackage{booktabs}
\usepackage{float}

\geometry{left=2.5cm,right=2.5cm,top=2.5cm,bottom=2.5cm}

\title{计算机图形学 作业一 实验报告}
\author{姓名:元朗曦 \\ 学号:23336294}
\date{2025 年 11 月 10 日}

\begin{document}

\maketitle

\section{实验目的}
\begin{enumerate}
    \item 熟悉 OpenGL 图形编程环境的搭建与配置(CMake, GLFW, GLM, ImGui)。
    \item 掌握 OpenGL 基本图元(三角形、四边形等)的绘制方法与特性。
    \item 理解并掌握二维与三维坐标变换、观察矩阵及投影矩阵(正交投影与透视投影)的原理与实现。
    \item 实现三维物体的建模(拉伸)及交互式旋转控制。
\end{enumerate}

\section{实验原理}
\subsection{基本图元}
OpenGL 提供了多种图元绘制模式。本实验主要涉及:
\begin{itemize}
    \item \textbf{GL\_TRIANGLES}: 每三个顶点定义一个独立的三角形。
    \item \textbf{GL\_TRIANGLE\_STRIP}: 复用前两个顶点,每增加一个顶点定义一个新三角形,效率较高。
    \item \textbf{GL\_QUADS}: 每四个顶点定义一个四边形(在现代 Core Profile 中已废弃,但在兼容模式下可用)。
\end{itemize}

\subsection{坐标变换与投影}
图形渲染管线中,顶点坐标经过模型变换(Model)、视图变换(View)和投影变换(Projection)最终映射到屏幕空间。
\begin{itemize}
    \item \textbf{正交投影 (Orthographic)}: 视景体为长方体,物体大小不随距离改变,保留平行性。
    \item \textbf{透视投影 (Perspective)}: 视景体为平截头体,物体遵循“近大远小”规律,模拟人眼视觉。
\end{itemize}

\section{实验步骤}

\subsection{环境搭建}
\subsubsection{开发环境}
\begin{itemize}
    \item \textbf{操作系统}: Windows
    \item \textbf{IDE}: Visual Studio Code
    \item \textbf{构建工具}: CMake (3.x+)
    \item \textbf{编译器}: MSVC (Visual Studio Build Tools)
    \item \textbf{依赖库}: GLFW (窗口), GLM (数学), Dear ImGui (UI)。
\end{itemize}

\subsubsection{搭建过程}
本项目采用 \texttt{CMake} 的 \texttt{FetchContent} 模块进行依赖管理。
\begin{enumerate}
    \item 配置 \texttt{CMakeLists.txt},使用 \texttt{FetchContent\_MakeAvailable} 引入依赖。
    \item 解决 GLM 包含路径及 ImGui OpenGL 后端绑定的问题。
\end{enumerate}

\subsection{绘制平面姓名首字母}
\subsubsection{实现思路}
\begin{itemize}
    \item \textbf{姓名首字母}: Y, L, X。
    \item \textbf{绘制方法}: 定义 \texttt{drawQuad} 函数,根据图元模式绘制四边形。
    \item \textbf{Y}: 3 个矩形拼接;\textbf{L}: 2 个矩形拼接;\textbf{X}: 2 个交叉四边形。
    \item 使用 \texttt{glTranslatef} 将各字母平移至以原点为中心的排列位置。
\end{itemize}

\begin{figure}[H]
    \centering
    \includegraphics[width=0.8\textwidth]{assets/images/2d-ylx.png}
    \caption{二维姓名首字母绘制结果 (YLX)}
\end{figure}

\subsection{绘制立体姓氏首字母}
\subsubsection{实现思路}
\begin{itemize}
    \item \textbf{立体化 (Extrusion)}: 实现 \texttt{drawExtrudedQuad},在 Z 轴方向延伸 2D 顶点,绘制前、后及四个侧面。
    \item \textbf{深度测试}: 开启 \texttt{GL\_DEPTH\_TEST}。
    \item \textbf{旋转实现}: 使用 \texttt{glRotatef} 配合 ImGui 提供的交互参数(自动/手动旋转、旋转轴选择)实现模型旋转。
\end{itemize}

\begin{figure}[H]
    \centering
    \includegraphics[width=0.8\textwidth]{assets/images/3d-y.png}
    \caption{立体姓氏首字母 Y}
\end{figure}

\section{实验总结}

\subsection{图元绘制开销比较}
通过 \texttt{vertexCount} 统计,绘制 YLX (共 7 个四边形) 的开销如下:

\begin{table}[H]
    \centering
    \begin{tabular}{|l|c|c|}
        \hline
        \textbf{图元类型} & \textbf{单四边形顶点数} & \textbf{总顶点数} \\
        \hline
        \textbf{GL\_TRIANGLES} & 6 & 42 \\
        \hline
        \textbf{GL\_TRIANGLE\_STRIP} & 4 & 28 \\
        \hline
        \textbf{GL\_QUADS} & 4 & 28 \\
        \hline
    \end{tabular}
    \caption{图元绘制开销比较}
\end{table}
\textbf{分析}: \texttt{GL\_TRIANGLE\_STRIP} 和 \texttt{GL\_QUADS} 相比 \texttt{GL\_TRIANGLES} 节省了约 33\% 的顶点传输量。

\subsection{引用说明}
本作业使用了 GLFW, GLM, Dear ImGui 开源库。核心绘制逻辑(字母定义、立体拉伸、交互控制)均为原创实现。

\end{document}